%-----------------------------------------------------------------------------
%
%               Template for sigplanconf LaTeX Class
%
% Name:         sigplanconf-template.tex
%
% Purpose:      A template for sigplanconf.cls, which is a LaTeX 2e class
%               file for SIGPLAN conference proceedings.
%
% Guide:        Refer to "Author's Guide to the ACM SIGPLAN Class,"
%               sigplanconf-guide.pdf
%
% Author:       Paul C. Anagnostopoulos
%               Windfall Software
%               978 371-2316
%               paul@windfall.com
%
% Created:      15 February 2005
%
%-----------------------------------------------------------------------------


\documentclass{sigplanconf}

% The following \documentclass options may be useful:

% preprint      Remove this option only once the paper is in final form.
% 10pt          To set in 10-point type instead of 9-point.
% 11pt          To set in 11-point type instead of 9-point.
% authoryear    To obtain author/year citation style instead of numeric.

\usepackage{amsmath}
\usepackage{amssymb}
\usepackage{stmaryrd}
\usepackage{upgreek}
\usepackage{pig}

\newcommand{\hb}{\!:\!}
\newcommand{\lol}{\multimap}
\newcommand{\nts}[1]{\marginpar{\tiny #1}}
\newcommand{\stk}[1]{\begin{array}{@{}c@{}}#1\end{array}}
\newcommand{\stkl}[1]{\begin{array}{@{}l@{}}#1\end{array}}

\newtheorem{lem}{Lemma}
\newtheorem{defn}[lem]{Definition}
\newtheorem{thm}[lem]{Theorem}
\newtheorem{cor}[lem]{Corollary}
\newtheorem{fail}[lem]{Failure}

\begin{document}

\special{papersize=8.5in,11in}
\setlength{\pdfpageheight}{\paperheight}
\setlength{\pdfpagewidth}{\paperwidth}

\conferenceinfo{CONF 'yy}{Month d--d, 20yy, City, ST, Country} 
\copyrightyear{20yy} 
\copyrightdata{978-1-nnnn-nnnn-n/yy/mm} 
\doi{nnnnnnn.nnnnnnn}

% Uncomment one of the following two, if you are not going for the 
% traditional copyright transfer agreement.

%\exclusivelicense                % ACM gets exclusive license to publish, 
                                  % you retain copyright

%\permissiontopublish             % ACM gets nonexclusive license to publish
                                  % (paid open-access papers, 
                                  % short abstracts)

\titlebanner{banner above paper title}        % These are ignored unless
\preprintfooter{short description of paper}   % 'preprint' option specified.

\title{Quadratisch. Praktisch. Gut.}
%\subtitle{Subtitle Text, if any}

\authorinfo{Witheld}
           {Witheld}
           {witheld}

\maketitle

\begin{abstract}
\end{abstract}

\category{D.3.2}{Programming Languages}{Formal Definitions and Theory}

% general terms are not compulsory anymore, 
% you may leave them out
%\terms
%term1, term2

\keywords
dependent types, cubical sets

\section{Introduction}

We define a dependent type theory, where values (including types, of
course) can be joined by \emph{paths}, conceptualised as continuous
functions over the interval $[0,1]$. This type theory has the
following properties:
\begin{description}
\item[canonicity] closed expressions compute to canonical values;
\item[substitutivity] paths between values yield
  computational coercions between types which depend on those values;
\item[decidability] type checking is algorithmic;
\item[extensionality] the notion of path between
  functions is pointwise.
\end{description}

The definition is similar in spirit to the \emph{Observational
Type Theory} of Altenkirch, McBride and
Swierstra~\cite{DBLP:conf/plpv/AltenkirchMS07},
but its construction is greatly simplified by the adoption of
technology from the \emph{Cubical Type Theory} of Bezem, T. Coquand
and Huber~\cite{DBLP:conf/types/BezemCH13}.

By careful engineering, the definition we present has the property
of \textbf{proof erasure}, meaning that closed expressions can be
given a run time behaviour which does not rely in any way on the
data contained in paths. However, this property is
fragile in that it relies on the choice to connect types and
values only structurally.

Our type theory does \emph{not} exhibit the either of the following
mutually incompatible properties:
\begin{description}
\item[proof obliviousness] all paths between the same endpoints are
  judgmentally equal (an extreme form of the `K axiom', better known
  as `uniqueness of identity proofs');
\item[univalence] arbitrary computational equivalences between types induce paths.
\end{description}
Rather, we stand at the crossroads between the two: we can extend our
system either by internalizing the choice that paths can only be
structural, doubling down on proof erasure, or by destroying that
choice with additional paths containing nontrivial data thus
rejecting proof erasure for a run time computational system which
computes the representation changes which implement type equivalences.


\section{Bidirectional Type Theory, Still in Eden}

\newcommand{\gs}{\;\mid\;}
\newcommand{\neu}{\underline}
\newcommand{\TYPE}{\bigstar}
\newcommand{\EC}{\mathcal{E}}

The syntax of our type theory has sorts (which are types of types)
including a top sort $\TYPE$,
terms (whose types are checked) embedding eliminations (whose types are
synthesized) embedding variables in turn. Variables are assigned types by contexts.
\[\begin{array}{rrl}
\star & ::= & \TYPE \gs \ldots \\
s,t,P,R,S,T & ::= & \star \gs \neu e \gs \ldots \\
e,f,E,F & ::= & x \gs s\hb S \gs \ldots \\
\Gamma & ::= & \EC \gs \Gamma,x\hb S \gs \ldots
\end{array}\]

The typing rules are presented via three judgment forms: checking,
synthesis and subtyping. We omit contexts from typing rules, adopting
the convention that the context at a rule's conclusion is distributed
to all its premises. We thus need only write the \emph{extension} of
a context with an entry for a fresh variable, $x\hb S\vdash \cdots$,
or a context entry lookup for an existing variable $\dashv x\hb S$.
This discipline not only saves space: it also helps us to ensure that
the type system is stable under substitution by ensuring that we
treat variables only in ways whch are compatible with how we treat
terms. Before we add any `features' to our type theory, we must have
at least the following.

\[\begin{array}{c}
\textsc{syn-var}\;
\Rule{\dashv x\hb S}
     {x\in S}
\qquad\qquad
\textsc{syn-cut}\;
\Rule{\TYPE\ni S\quad S\ni s}
     {s\hb S\in S}
\\
\textsc{chk-syn}\;
\Rule{e\in S\quad S\le T}
     {T\ni \neu{e}}
\\
\textsc{sub-refl}\;
\Axiom{T\le T}
\qquad\qquad
\textsc{sub-top}\;
\Axiom{\star \le \TYPE}
\end{array}\]

So, for variables we synthesize exactly the type assigned by the
context; for cuts, we synthesize exactly the type proposed in the
annotation (provided it is checkably a type and indeed the type of the
term). The reason why a term with a type annotation is called a `cut':
it is in some sense activated for computation by being labelled with
its type, indicating its functionality.
If we can synthesize the type of something, we may surely check
it by requiring the type we get to be a subtype of the type we need,
where subtyping is at the very least reflexive.

Terms are blessed with a contraction scheme: cut elimination. A cut
embedded directly back into the term syntax can do no computation and
thus needs no annotation.
\[
\textsc{red-}\upsilon\quad
\neu{t\hb T}\leadsto t
\]

As soon as we have computation, we need to know its impact on
typechecking. We have rules which allow us to precompute types before
checking and postcompute types after synthesis.
\[
\textsc{chk-pre}\;
\Rule{T\leadsto R \quad R\ni t}
     {T\ni t}
\qquad
\textsc{syn-post}\;
\Rule{e\in S \quad S\leadsto R}
     {e\in R}
\]

Let us add dependent pairs ($\Sigma$-types), by way of an example feature.
\[\begin{array}{c}
\textsc{chk-sg}\;
\Rule{\star\ni S\quad x\hb S\vdash \star\ni T[x]}
     {\star\ni (x\hb S)\times T[x]}
\qquad
\textsc{syn-1}\;
\Rule{e\in (x\hb S)\times T[x]}
     {e.1\in S}
\\
\textsc{chk-pair}\;
\Rule{S\ni s\quad T[s\hb S]\ni t}
     {(x\hb S)\times T[x]\ni s,t}
\qquad
\textsc{syn-2}\;
\Rule{e\in (x\hb S)\times T[x]}
     {e.2\in T[e.1]}
\\
\textsc{sub-sg}\;
\Rule{S\le S' \quad x\hb S\vdash T[x]\le T'[x]}
     {(x\hb S)\times T[x]\le(x\hb S')\times T'[x]}
\end{array}\]
Every sort is closed under the formation of
$\Sigma$-types. $\Sigma$-types
admit pairs, where the first component, once checked, is used to
compute the type of the second. By putting introduction forms in
the terms and elimination forms in the eliminations (hence the
names), we ensure that introduction forms can only be eliminated
at a cut which exposes the type at which activity is present.
Computation, indeed, goes by $\beta$-rules.
\[\begin{array}{c}
\textsc{red-$\beta$-1}\quad 
  (s,t:(x\hb S)\times T[x]).1 \leadsto s:S \\
\textsc{red-$\beta$-2}\quad 
  (s,t:(x\hb S)\times T[x]).2 \leadsto t:T[s\hb S]
\end{array}
\]
So, a computation labelled as being at a compound type `decays' to
subcomputations labelled at the components (or instances of them).
Cuts bubble outwards through the elimination syntax, decaying as they go,
until they reach the point where the result is embedded in a term,
whereupon the cut can be eliminated.

For function types, we see a similar pattern.
\[\begin{array}{c}
\textsc{chk-pi}\;
\Rule{\star\ni S\quad x\hb S\vdash \star\ni T[x]}
     {\star\ni (x\hb S)\to T[x]}
\\
\textsc{chk-lam}\;
\Rule{x\hb S\vdash T[x]\ni t[x]}
     {(x\hb S)\to T[x]\ni \lambda x.\,t[x]}
\\
\textsc{syn-app}\;
\Rule{f\in (x\hb S)\times T[x]\quad S\ni s}
     {f\:s \in T[s\hb S]}
\\
\textsc{sub-pi}\;
\Rule{S'\le S \quad x\hb S'\vdash T[x]\le T'[x]}
     {(x\hb S)\to T[x]\le(x\hb S')\to T'[x]}
\\
\textsc{red-}\beta\textsc{-app}\quad
(\lambda x.\,t[x]:(x\hb S)\times T[x])\:s \leadsto t[s\hb S]:T[s\hb S]
\end{array}\]

Let us add Booleans as a base type in every sort, but with large elimination.
\newcommand{\Two}{2}
\newcommand{\true}{\mathsf{t\!t}}
\newcommand{\false}{\mathsf{f\!f}}
\newcommand{\iffy}[4]{(\textsc{if}\:(#1.\,#2)\:#3\:#4)}
\[\begin{array}{c}
\textsc{chk-two}\;
\Axiom{\star\ni \Two}\qquad
\textsc{chk-tt}\;
\Axiom{\Two\ni \true}\qquad
\textsc{chk-ff}\;
\Axiom{\Two\ni \false}\\
\textsc{syn-if}\;
\Rule{e\in\Two\quad b\hb\Two\vdash \TYPE\ni P[b]\quad
      P[\true\hb\Two]\ni t\quad P[\false\hb\Two]\ni f}
     {e\:\iffy{b}{P[b]}{t}{f}\in P[e]}\\
\textsc{red-}\beta\textsc{-if-tt}\quad
(\true\hb\Two)\:\iffy{b}{P[b]}{t}{f} \leadsto t\hb P[\true\hb\Two] \\
\textsc{red-}\beta\textsc{-if-ff}\quad
(\false\hb\Two)\:\iffy{b}{P[b]}{t}{f} \leadsto f\hb P[\false\hb\Two] \\
\end{array}\]

We can also have empty and unit types.
\newcommand{\Zero}{\mathsf{0}}
\newcommand{\One}{\mathsf{1}}
\newcommand{\void}{\langle\rangle}
\newcommand{\magic}[1]{(\textsc{magic}\:#1)}
\[\begin{array}{c}
\textsc{chk-zero}\;
\Axiom{\star\ni \Zero}\qquad
\textsc{chk-one}\;
\Axiom{\star\ni \One}\\
\textsc{syn-magic}\;
\Rule{e\in\Zero\quad \TYPE\ni P}
     {e\:\magic{P}\in P}\qquad
\textsc{chk-void}\;
\Axiom{\One\ni \void}\\
\end{array}\]
We could add a dependent eliminator for $\One$, asserting that every
element is $\void$, or we could just add an $\eta$-rule.

\newcommand{\Type}[1]{\star^{#1}}

\textbf{datatypes of some sort?}

\newcommand{\jq}{\cong}
\textbf{introduce the judgmental equality} $\jq$


\section{Paths As Abstractions Over the Interval}

The term `equality' is laden with philosophical baggage. Let us slough
off that burden and consider the idea of connecting things with paths.
There might very well be information in a path connecting things that
is not already in the things themselves: for example, a road route
connecting Geneva and Montreux will have to go around Lake Geneva one
way or the other. Of course, we shall need to know how to construct
paths and how we expect to be able to use them.

\newcommand{\pl}{0}
\newcommand{\pr}{1}
The key mechanism of cubical type theories is the idea of constructing
paths in a pointwise manner: we allow a parameter $i$ to vary
`continuously' between $\pl$ and $\pr$, and we give a construction
$t[i]$ valid at all $i$, thus exposing a connection between $t[\pl]$
and $t[\pr]$. Of course, paths are constructions, too, so we can form
paths-between-paths by abstracting over one parameter a construction
which then abstracts over a second: some $t[i,j]$ varying over a
square. Paths-between-paths-between-paths abstract over cubes, and so
on up the dimensions, hence `cubical' type theory.

Let us start with paths between types.
\newcommand{\tp}[1]{\frac{#1}{\;\;\;}}
\[\begin{array}{c}
\textsc{chk-typ}\;
\Rule{\star\le\star'\quad \star \ni S\quad\star\ni T}
     {\star' \ni S\tp{\star}T}
\\
\textsc{chk-typ-lam}\;
\Rule{\bar i\vdash \star \ni P[i]\quad S\jq P[\pl]\quad P[\pr]\jq T}
     {S\tp{\star}T \ni \lambda i.\,P[i]}
\end{array}\]
Note that the interval between $\pl$ and $\pr$ is not a type, so the
binding in context $\bar i$ is of a different form. We shall need a
separate judgment for \emph{points}.
\newcommand{\pt}[1]{\textsc{pt}\;#1}
\[
\textsc{pt-$\pl$}\;\Axiom{\pt\pl}\qquad 
\textsc{pt-$\pr$}\;\Axiom{\pt\pr}\qquad
\textsc{pt-var}\;\Rule{\dashv \bar i}
                      {\pt i}
\]
Now we can explain how to project from a path at a given
point.
\[\begin{array}{c}
\textsc{syn-typ-app}\;
\Rule{E\in S\tp{\star}T\quad \pt p}
     {E\:p\in\star}
\\
\textsc{red-typ-$\beta$}\;
(\lambda i.\,P[i]:S\tp{\star}T)\:p\leadsto P[p]:\star
\end{array}\]

It is usual to add further computation rules, allowing projection
of \emph{endpoints}, necessitating type reconstruction.
\[\begin{array}{c}
\textsc{red-typ-$\pl$?}\;
  E\in S\tp\star T \;\Rightarrow\; E\:\pl\leadsto S\hb\star\\
\textsc{red-typ-$\pr$?}\;
  E\in S\tp\star T \;\Rightarrow\; E\:\pr\leadsto T\hb\star\\
\end{array}\]
The critical pairs with $\beta$ induced by these rules are
guaranteed to be resolvable, given that the construction of canonical
paths enforces coincidence with the endpoints stated in the type.

%\appendix
%\section{Appendix Title}
%
%This is the text of the appendix, if you need one.

\acks

Many acknowledgments are due. We have had a lot of assistance, much
inspiration from colleagues and many enlightening conversations. Names
will be named and honour bestowed if and when this paper ceases to be
anonymous.

% We recommend abbrvnat bibliography style.

\bibliographystyle{abbrvnat}

% The bibliography should be embedded for final submission.

\bibliography{Crossroads}

%\begin{thebibliography}{}
%\softraggedright
%
%\bibitem[Smith et~al.(2009)Smith, Jones]{smith02}
%P. Q. Smith, and X. Y. Jones. ...reference text...
%
%\end{thebibliography}


\end{document}

%                       Revision History
%                       -------- -------
%  Date         Person  Ver.    Change
%  ----         ------  ----    ------

%  2013.06.29   TU      0.1--4  comments on permission/copyright notices

