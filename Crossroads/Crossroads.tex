%-----------------------------------------------------------------------------
%
%               Template for sigplanconf LaTeX Class
%
% Name:         sigplanconf-template.tex
%
% Purpose:      A template for sigplanconf.cls, which is a LaTeX 2e class
%               file for SIGPLAN conference proceedings.
%
% Guide:        Refer to "Author's Guide to the ACM SIGPLAN Class,"
%               sigplanconf-guide.pdf
%
% Author:       Paul C. Anagnostopoulos
%               Windfall Software
%               978 371-2316
%               paul@windfall.com
%
% Created:      15 February 2005
%
%-----------------------------------------------------------------------------


\documentclass{sigplanconf}

% The following \documentclass options may be useful:

% preprint      Remove this option only once the paper is in final form.
% 10pt          To set in 10-point type instead of 9-point.
% 11pt          To set in 11-point type instead of 9-point.
% authoryear    To obtain author/year citation style instead of numeric.

\usepackage{amsmath}
\usepackage{amssymb}
\usepackage{stmaryrd}
\usepackage{upgreek}
\usepackage{pig}

\newcommand{\hb}{\!:\!}
\newcommand{\lol}{\multimap}
\newcommand{\nts}[1]{\marginpar{\tiny #1}}
\newcommand{\stk}[1]{\begin{array}{@{}c@{}}#1\end{array}}
\newcommand{\stkl}[1]{\begin{array}{@{}l@{}}#1\end{array}}

\newtheorem{lem}{Lemma}
\newtheorem{defn}[lem]{Definition}
\newtheorem{thm}[lem]{Theorem}
\newtheorem{cor}[lem]{Corollary}
\newtheorem{fail}[lem]{Failure}

\begin{document}

\special{papersize=8.5in,11in}
\setlength{\pdfpageheight}{\paperheight}
\setlength{\pdfpagewidth}{\paperwidth}

\conferenceinfo{CONF 'yy}{Month d--d, 20yy, City, ST, Country} 
\copyrightyear{20yy} 
\copyrightdata{978-1-nnnn-nnnn-n/yy/mm} 
\doi{nnnnnnn.nnnnnnn}

% Uncomment one of the following two, if you are not going for the 
% traditional copyright transfer agreement.

%\exclusivelicense                % ACM gets exclusive license to publish, 
                                  % you retain copyright

%\permissiontopublish             % ACM gets nonexclusive license to publish
                                  % (paid open-access papers, 
                                  % short abstracts)

\titlebanner{banner above paper title}        % These are ignored unless
\preprintfooter{short description of paper}   % 'preprint' option specified.

\title{A Cubical Crossroads}
%\subtitle{Subtitle Text, if any}

\authorinfo{Witheld}
           {Witheld}
           {witheld}

\maketitle

\begin{abstract}
\end{abstract}

\category{D.3.2}{Programming Languages}{Formal Definitions and Theory}

% general terms are not compulsory anymore, 
% you may leave them out
%\terms
%term1, term2

\keywords
dependent types, cubical sets

\section{Introduction}

We define a dependent type theory with the following properties:
\begin{description}
\item[canonicity] closed expressions compute to canonical values;
\item[substitutivity] paths between values yield
  computational coercions between types which depend on those values;
\item[decidability] type checking is algorithmic;
\item[extensionality] the notion of path between
  functions is pointwise.
\end{description}

The definition is similar in spirit to the \emph{Observational
Type Theory} of Altenkirch, McBride and
Swierstra~\cite{DBLP:conf/plpv/AltenkirchMS07},
but its construction is greatly simplified by the adoption of
technology from the \emph{Cubical Type Theory} of Bezem, T. Coquand
and Huber~\cite{DBLP:conf/types/BezemCH13}.

By careful engineering, the definition we present has the property
of \textbf{proof erasure}, meaning that closed expressions can be
given a run time behaviour which does not rely in any way on the
data contained in paths. However, this property is
fragile in that it relies on the choice to connect types and
values only structurally.

Our type theory does \emph{not} exhibit the either of the following
mutually incompatible properties:
\begin{description}
\item[proof obliviousness] all paths between the same endpoints are
  judgmentally equal (an extreme form of the `K axiom', better known
  as `uniqueness of identity proofs');
\item[univalence] arbitrary computational equivalences between types induce paths.
\end{description}
Rather, we stand at the crossroads between the two: we can extend our
system either by internalizing the choice that paths can only be
structural, doubling down on proof erasure, or by destroying that
choice with additional paths containing nontrivial data thus
rejecting proof erasure for a run time computational system which
computes the representation changes which implement type equivalences.


%\appendix
%\section{Appendix Title}
%
%This is the text of the appendix, if you need one.

\acks

Many acknowledgments are due. I have had a lot of assistance, much
inspiration from colleagues and many enlightening conversations. Names
will be named and honour bestowed if and when this paper ceases to be
anonymous.

% We recommend abbrvnat bibliography style.

\bibliographystyle{abbrvnat}

% The bibliography should be embedded for final submission.

\bibliography{Crossroads}

%\begin{thebibliography}{}
%\softraggedright
%
%\bibitem[Smith et~al.(2009)Smith, Jones]{smith02}
%P. Q. Smith, and X. Y. Jones. ...reference text...
%
%\end{thebibliography}


\end{document}

%                       Revision History
%                       -------- -------
%  Date         Person  Ver.    Change
%  ----         ------  ----    ------

%  2013.06.29   TU      0.1--4  comments on permission/copyright notices

