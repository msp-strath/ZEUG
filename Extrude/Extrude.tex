\documentclass{article}
%%\usepackage{fourier} %% \widearc
\usepackage{mathabx} %% \overgroup \wideparen
\usepackage{upgreek}
\usepackage{pig}

\ColourEpigram

\begin{document}
\title{Excuse My Extrusion}
\author{Conor McBride and some friends, probably}
\maketitle

\newcommand{\blob}{\orange{\bullet}}
\newcommand{\VP}[3]{#2\!\!\mathrel{\blue{\overgroup{\;\black{#1}\;}}}\!\!#3}

\newcommand{\stk}[1]{\begin{array}{@{}c@{}}#1\end{array}}

\section{Ordinary Things Before Cubes}

\newcommand{\TYPE}[1]{\textsc{type}\;#1}
\newcommand{\HAS}[2]{#1 \ni #2}
\newcommand{\IS}[2]{#1 \in #2}
\newcommand{\SUB}[2]{#1 \le #2}
\newcommand{\CX}[1]{\dashv #1}

\newcommand{\hb}{\!:\!}
\newcommand{\sbs}[2]{\lgroup #1/#2\rgroup}

\newcommand{\PI}[3]{\blue{(}#1\hb#2\blue{)} \mathrel{\blue{\to}} #3}
\newcommand{\LA}[2]{\red{\uplambda}\,#1 \mathrel{\red{\to}} #2}
\newcommand{\NE}{\underline}

\[\stk{
\Rule{\IS e S \quad\SUB S T}
     {\HAS{T}{\NE{e}}}
\qquad
\Rule{\TYPE S \quad\HAS S s}
     {\IS {s\hb S} S}
\qquad
\Rule{\CX{x\hb S}}
     {\IS x S}
\\
\Rule{\TYPE S \quad x\hb S \vdash \TYPE T}
     {\TYPE {\PI xST}}
\qquad
\Rule{x\hb S \vdash \HAS T t}
     {\HAS {\PI xST} {\LA xt}}
\qquad
\Rule{\IS f {\PI xST}\quad \HAS S s}
     {\IS {f\:s} {T\sbs{s\hb S}x}}
\\
\Rule{\SUB{S'} S \quad x\hb S'\vdash\SUB T {T'}}
     {\SUB{\PI xST}{\PI {x'}{S'}{T'}}}
}\]


\section{Points, Paths}

\newcommand{\SYM}[1]{{#1}^\degree}
\newcommand{\POINT}[1]{\textsc{point}\;#1}
\newcommand{\PL}{\red{0}}
\newcommand{\PR}{\red{1}}
\newcommand{\IA}[1]{\left< #1\right>}
\newcommand{\IL}[2]{\red{\left<\right.}\!#1\!\red{\left.\right>} #2}
\newcommand{\EQ}{\mathrel{\blue{=}}}

\[
\Axiom{\POINT \PL}
\qquad
\Axiom{\POINT \PR}
\qquad
\Rule{\dashv \IA{i}}
     {\POINT i}
\qquad
\Rule{\POINT p\quad \POINT p_0 \quad \POINT p_1}
     {\POINT p[p_0-p_1]}
\]

The evaluation rules for points are
\[
  \PL [p_0-p_1] \equiv p_0 \qquad \PR [p_0-p_1] \equiv p_1
\]

\newcommand{\pn}{\textsc{pnorm}}
The normal form of a point in a given context is computed as follows:
\[\begin{array}{lcl}
\pn\: \cdot\: \PL & \equiv & \PL \\
\pn\: \cdot\: \PR & \equiv & \PR \\
\pn\: \cdot\: (\PL [p_0-p_1]) & \equiv & \pn\: \cdot\: p_0 \\
\pn\: \cdot\: (\PR [p_0-p_1]) & \equiv & \pn\: \cdot\: p_1 \\
\pn\: (\Gamma,x\hb S)\: p   & \equiv & \pn\: \Gamma\: p \\
\pn\: (\Gamma,\IA{i})\: p   & \equiv & \mbox{if}\;j\equiv k\;\mbox{then}\;j\;\mbox{else}\;i [j-k]\\
 && \mbox{where}\; j\equiv\pn\:\Gamma\: p\sbs \PL i;\; k\equiv \pn\:\Gamma\: p\sbs \PR i
\end{array}\]

For display purposes, it is polite to write $i[\PL-\PR]$ as $i$ and $i[\PR-\PL]$ as $\SYM i$.

Type path types are inhabited by abstraction over the interval $\IL i R$, giving a type which deforms
continuously over the interval, coinciding with the left end required when $i\equiv\PL$ and the right end
when $i\equiv\PR$.
\[
\Rule{\TYPE S \quad \TYPE T}
     {\TYPE {S \EQ T}}
\qquad
\Rule{\IA{i} \vdash \TYPE R \qquad S\equiv R\sbs \PL i \qquad R\sbs \PR i \equiv T}
     {\HAS {S \EQ T}{\IL i R}}
\qquad
\Rule{\IS Q {S \EQ T}\quad \POINT p}
     {\TYPE {\NE{Q\:p}}}
\]

Value paths are a little more fiddly. Firstly, one must form them `over' type paths which establish the basis for making a connection between values. Secondly, it is not compulsory to choose the endpoints. The symbol $\blob$ (`blob') stands for an unspecified endpoint in a value path type. I also overload it as the predicate $\blob \cdot$ which tests whether an endpoint is just a blob. Endpoints, if specified, must have the types given at the endpoints of the type path.
\[
\Rule{\IS Q {S \EQ T}\quad \vee\stk{\blob s \\ \HAS S s} \quad
                           \vee\stk{\blob t \\ \HAS T t}}
     {\TYPE {\VP Q s t}}
\]

To construct a value path, it is again necessary to abstract over the interval and construct a value which fits the type path at each point on it. If endpoints are specified, the abstracted value must coincide with them at the relevant end.
\[
\Rule{\IA{i} \vdash \HAS {\NE{Q\:i}} r \quad
      \vee\stk{\blob s \\ \HAS {\NE{Q\:\PL}} {s\equiv r\sbs \PL i}} \quad 
      \vee\stk{\blob t \\ \HAS {\NE{Q\:\PR}} {r\sbs \PR i \equiv t}}}
     {\HAS {\VP Q s t} {\IL i r}}
\qquad
\Rule{\IS q {\VP Q s t}\quad \POINT p}
     {\IS {q\:p} {\NE{Q\:p}}}
\]

We can think of value paths as subtypes in a lattice, in accordance with their specificity of endpoints.
(Is this subtyping really necessary?) 
\[
\Axiom{\SUB {\VP Q s \blob} {\VP Q \blob \blob}}
\qquad
\Axiom{\SUB {\VP Q \blob t} {\VP Q \blob \blob}}
\qquad
\Axiom{\SUB {\VP Q s t} {\VP Q s \blob}}
\qquad
\Axiom{\SUB {\VP Q s t} {\VP Q \blob t}}
\]

Let us adopt some convenient shorthands. Choosing bound variables to avoid capture (as is trivial in a de Bruijn implementation),
\[
Q[p-p'] \equiv \IL i {Q\:i[p-p']}
\qquad
\SYM Q \equiv Q[1-0]
\]

Weakening, we readily have that constant paths give reflexivity. Let us write $\IL {} t$ to emphasize that there is no use of the bound variable. Derivably,
\[
\Rule{\TYPE{T}}
     {\HAS {T \EQ T}{\IL {} T}}
\qquad
\Rule{\HAS T t}
     {\HAS {\VP {\IL {} T} t t} {\IL {} t}}
\]



\section{Extrusion}

\newcommand{\EXR}{\mathop{\green{\raisebox{0.03in}{$\frown$}\!\!\bullet}}}
\newcommand{\EXL}{\mathop{\green{\bullet \!\!\raisebox{0.03in}{$\frown$}}}}

\[
\Rule{\IS Q {S \EQ T} \qquad \HAS S s}
     {\IS {s \EXR Q} {\VP Q s \blob}}
\]

For cosmetic purposes, let us define leftward extrusion:
\[
\Rule{\IS Q {S \EQ T} \qquad \HAS T t}
     {\IS {Q \EXL t} {\VP Q \blob t}}
\qquad
Q \EXL t \equiv \SYM{(t \EXR \SYM Q)}
\]


\newcommand{\PA}[1]{\lgroup #1 \rgroup}

We need to arrange for extrusion to compute for all the types we can
encounter. Let us start with functions. I write $S\PA i$, etc, marking
explicitly the dependency on $i$, in order to abbreviate substitutions.
I further write $S\PA{p-p'} \equiv \IL j {S\PA{j[p-p']}}$.

\[
f\EXR \IL i {\PI x {S\PA{i}} {T\PA{i,x}}} \equiv
  \IL i {\LA x {\begin{array}[t]{@{}l@{\;}l@{\,}c@{\,}l@{}}
         \mathbf{let} & \bar x & \equiv & S\PA{\PL-i} \EXL x \\
                      & \bar y & \equiv & f\:(\bar x\:\PL)
                                      \EXR \IL j {T\PA{j[\PL-i],\bar x\:j}} \\
         \mathbf{in}  & \multicolumn{3}{@{}l}{\bar y\:\PR}
         \end{array}}}
\]

Why does this typecheck? Well,
\[
\bar{y}\:1 \in T\PA{i,\bar{x}\:1} \equiv T\PA{i,x}
\]
so the typical point on the value path fits the typical point on the type path.
But we must also check the left end, taking $i\equiv\PL$.
\[
  (f \EXR \IL i {\PI x {S\PA{i}} {T\PA{i,x}}})\:\PL \quad
\mbox{makes}\quad\begin{array}[t]{@{}l@{\,}c@{\,}l}
  \bar x &\equiv& (\IL j {S\PA\PL}) \EXL x \\
  \bar y &\equiv& f\:(\bar{x}\:\PL)  \EXR (\IL j {T\PA{\PL,\bar{x}\:j}})
  \end{array}
\]
So, if we can arrange for extrusion over constant type paths to produce constant
value paths, we shall have that $\bar x\:j\equiv x$ and thence that
$\bar y\:j\equiv f\:x$. The left end of the extrusion thus transports $f$ to its
$\eta$-expansion, so let us ensure that the $\eta$-laws hold.

The (contentious) suggestion is that the following should hold judgmentally.
\[
\HAS{\VP {\IL{}T} t \blob}{t\EXR\IL{}T \equiv \IL{}t}
\]


Let us now consider how to extrude across path types. What is
\[
Q \EXR \IL i {S\PA i \EQ T\PA i} \equiv ?
\]
? We have $Q \in S\PA\PL \EQ T\PA\PL$, and of course we can form paths
$S\PA{i-\PL}$ and $T\PA{\PL-i}$, but can we compose these three?


\section{Transitivity as Composition}

We don't get transitivity for paths for free from substitutivity, because
substitutivity is derived from extrusion and extrusion is what we are trying to
make compute. We shall have to add a composition operator.
\newcommand{\PC}{\mathop{\red{\circ}}}
\[
\Rule{\IS {Q_0} {R\EQ S_0} \quad \IS {Q_1} {S_1\EQ T}\quad S_0\equiv S_1}
     {\IS {Q_0\PC Q_1} {R \EQ T}}
\qquad
\stk{
(Q_0\PC Q_1)\:\PL \equiv Q_0\:\PL \\
(Q_0\PC Q_1)\:\PR \equiv Q_1\:\PR }
\]

Of course, we shall also need composition for value paths.
\[
\Rule{\IS {q_0} {\VP {Q_0} r {s_0}}\quad \IS {q_1} {\VP {Q_1} {s_1} t} \quad
      Q_0\:\PR\equiv Q_1\:\PL \quad \HAS {Q_0\:\PR} {q_0\:\PR \equiv q_1\:\PL}}
     {\IS {q_0} {\VP {Q_0\PC Q_1} r t}}
\]
with the same computation rules.

Now we can define extrusion across a composition.
\[
r\EXR{(Q_0\PC Q_1)} \equiv
  \begin{array}[t]{@{}l@{\;}l@{\,}c@{\,}l@{}}
      \mathbf{let} & {\bar r}_0 & \equiv & r \EXR Q_0 \\
                   & {\bar r}_1 & \equiv & {\bar r}_0\:\PR \EXR Q_1 \\
      \mathbf{in}  & \multicolumn{3}{@{}l}{{\bar r}_0\PC {\bar r}_1}
  \end{array}
\]

We can also finish the definition of extrusion across a type path type.
\[
Q \EXR \IL i {S\PA i \EQ T\PA i} \equiv S\PA{i-\PL}\PC Q \PC T\PA{\PL-i}
\]

And we had better give the counterpart for value path types.
\[
q \EXR \IL i {\VP {Q\PA i} {s\PA{i}} {t\PA{i}}} \equiv
  \begin{array}[t]{@{}l@{\;}l@{\,}c@{\,}l@{}}
      \mathbf{let} & {\bar s} & \equiv & \mbox{if}\;\blob s\;
                           \mbox{then}\; Q\PA{i-\PL}\:\PL \EXL q\:\PL\;\mbox{else}\;s\PA{i-0} \\
                   & {\bar t} & \equiv & \mbox{if}\;\blob t\;
                           \mbox{then}\; q\:\PR \EXR Q\PA{\PL-i}\:\PR \;\mbox{else}\;t\PA{0-i} \\
      \mathbf{in}  & \multicolumn{3}{@{}l}{{\bar s}\PC q \PC {\bar t}}
  \end{array}
\]

What is the equational theory of composition? Is it just a plain constructor?
I think not. It has laws which we must somehow respect.
\[
\HAS {S\EQ T}{(\IL {} S : S\EQ S)\PC Q \equiv Q} \qquad
\HAS {S\EQ T}{Q\PC(\IL {} T : T\EQ T)\PC Q \equiv Q}
\]
and similarly for value paths. Also, composition should be associative.

Moreover, composition should commute with canonical constructors.
\[\begin{array}{ll}
 & (\IL{i}{\PI x {S\PA{i}}{T\PA{i,x}}})\PC
    (\IL{i}{\PI x {S'\PA{i}}{T'\PA{i,x}}}) \\
  \equiv &
  \IL{i}{\PI x {S\PA{0-1}\PC S'\PA{0-1}\:i}{(\IL{j}{T\PA{j,?}})\PC (\IL{j}{T'\PA{j,?}})\:i}}
\end{array}\]
Ah. There's a problem even with \emph{stating} this property. To see why,
consider the problem
\[
  \IA{i},x\hb S\PA{0-1}\PC S'\PA{0-1}\:i \vdash \TYPE ?
\]
We need to build a composite dependent type from $T\PA{-,-}$ and
$T'\PA{-,-}$ which agrees with $x$ at every point $i$. Near $\PL$, $x$
belongs to $S\PA{-}$, so we need to give $T\PA{-,-}$, but at whatever
notional point we weld $S\PA{\PR}$ to $S'\PA{\PL}$, we need to switch
(continuously), from $T\PA{\PR,x}$ to $T'\PA{\PL,x}$. We do not have a
way to talk about that point, or to build dependent things by
composition at a specific point.

It seems that we have two options: do without, or figure out how to
engineer correspondence at internal points in paths.


\section{Refining Path Types}

I had a few ideas about path types and internal points. I don't know
if we want to go this way, but I thought I'd write them down before I
forget them.

The key is to consider $\EQ$ to be the type former for the simplest
notion of path, but to allow more complex notions, with internal
points. We need to define valid path forms and generalise path types.
\newcommand{\TPATH}{\textsc{tpath}\:}
\newcommand{\TP}[1]{\mathrel{\blue{\left[\black{#1}\right]}}}
\[
\Axiom{\TPATH \EQ}\qquad
\Rule{\TPATH\pi\quad \TYPE S\quad\TPATH\rho}
     {\TPATH\TP{\pi S \rho}} \qquad
\Rule{\TYPE R\quad \TPATH\pi \quad \TYPE T}
     {\TYPE {R\pi T}}
\]
\newcommand{\PP}{\red{\cdot}}
The idea is that a path type $R\TP{\EQ S \EQ}T$ meets $R$ at $\PL$ and
$T$ at $\PR$, but also has a \emph{midpoint}, $\PP$, where it meets
$S$. There's an obvious notion of refinement for paths, inducing
subtyping of path types.
\[
\Axiom{\TP{\pi S \rho} \le \EQ}\qquad
\Rule{\pi\le\pi'\quad\rho\le\rho'}
     {\TP{\pi S \rho} \le \TP{\pi' S \rho'}} \qquad
\Rule{\pi\le\rho}
  {R\pi T \le R\rho T}
\]
Each refinement of $\EQ$ puts a midpoint in a segment, so the point
values can sensible represented as binary fractions between $\PL$ and
$\PR$, written with an initial binary point and omitting the last
$\PR$. So $\PP$ is a half, $\PP\PL$ a quarter, $\PP\PR$ three
quarters, and so on: the bits tell you whether to step left or right
until the segment whose midpoint is being selected.
\newcommand{\LP}[2]{\mathrel{\red{\left[\black{#1}\bullet\black{#2}\right]}}}
The corresponding introduction form also describes the segment structure
of a path. (Why? See below.) Segment structures are just binary trees. The blank
segment structure has no internal points. It's not hard to see when
a segment structure refines a path type.
\newcommand{\SEG}{\textsc{lpath}\:}
\[
\Axiom{\SEG}\qquad
\Rule{\SEG p\quad\SEG r}
     {\SEG \LP p r}\qquad
\Axiom{p \preceq \EQ}\qquad
\Rule{p\preceq\pi\quad r\preceq\rho}
     {\LP p r \preceq \TP{\pi S \rho}}
\]
Now we may generalise the interval abstraction to allow more
information.
\[
\Rule{p\preceq \pi\quad \IA{i\:p} \vdash \TYPE S}
     {\HAS{R\pi T}{\IL{i\:p} S}}
\]
The point is this: even if the path type does not specify particular
internal points, we can nonetheless construct a path in segments. The
path type identifies a minimal collection of internal points which
must exist in any such path, but a path construction may have more.
If we are to justify the refinement subtyping for path types, we must
be sure that constructions which use greater detail can still be
accessed with less detail.

(What is the interaction with rescaling?)

The segment structure gives a rigid framework of internal points,
allowing coherent constructions, but subtyping throws that rigidity
away, after the fact. Judgmental equality is typed, and the tighter
the subtype, the less forgiving the equality: the additional segment
structure of paths unspecified by type is flattened. What we do know
is that an interval variable with nontrivial segment structure can be
substituted by its midpoint. We can perform that move at any point
in a construction.
\newcommand{\JC}[1]{\mathrel{\red{-\!\langle\black{#1}\rangle\!-}}}
\newcommand{\LH}{\green{\triangleleft\!\!\bullet}}
\newcommand{\RH}{\green{\bullet\!\!\triangleright}}
\[
\Rule{\Gamma,\IA{i\:p},\Delta\PA{\LH}\vdash \TYPE S\PA i \quad
      \Gamma,\IA{i\:r},\Delta\PA{\RH}\vdash \TYPE T\PA i \quad
      \Gamma,\Delta\PA{\PP} \vdash S\PA\PR \equiv T\PA\PL}
     {\Gamma,\IA{i\:\LP p r},\Delta\PA{i}\vdash
        \TYPE{S\PA i\JC i T\PA i}}
\]
This rule slices space in two and rescales the pieces to run from
$\PL$ to $\PR$. Crucially, we have to be able to focus separately
on the left part and the right part of the interval, which is what
the segment selections $\LH$ and $\RH$ do.
\[
  (S\JC i T)\PA{\LH/i} \equiv S\qquad
  (S\JC i T)\PA{\RH/i} \equiv T
\]
(But what about the rest of it? What about rescaling, \emph{indeed}?)
We should make sure that we can always distribute $\JC i$ out as far
as we like, up to the abstraction which introduces $i$. The segment selection
behaviour should amount to distribution followed by projection of the
left or right part. Note that
\[
  (S\PA i\JC i T\PA i)\PA{\PP/i} \equiv S\PA\PR \equiv T\PA\PL
\]

Path composition is now derivable, with a more exciting type
\[
Q\PC Q' \equiv (\IL{i\:\LP{}{}} Q\:i\JC i Q'\:i :
               Q\:\PL\TP{\EQ Q\:\PR\EQ}Q'\:\PR)
\]
with $Q'\:\PL$ also permitted in the middle. That's to say, the
composition is constructed in two halves, and we are not obliged to
forget what is at the midpoint. Subtyping recovers the original type
of more plastic compositions.

(Rescaling: does it just match the segment structure? If so, can we
have loopy paths?)

We should be able to project the segments from a path.

(What's the value-path story?)

But I digress. There's a power-to-weight calculation needed here.



\section{Sigma Types}
\newcommand{\SG}[3]{\blue{(}#1\hb#2\blue{)} \mathrel{\blue{\times}} #3}
\newcommand{\pr}{\red{,}\:}
\newcommand{\car}{\green{\textsc{car}}\:}
\newcommand{\cdr}{\green{\textsc{cdr}}\:}

\[
\Rule{\TYPE S \quad x\hb S \vdash \TYPE T}
     {\TYPE {\SG xST}}
\qquad
\Rule{\HAS S s \quad \HAS{T\sbs{s\hb S}x}t}
     {\HAS {\SG xST}{s\pr t}}
\]
\[
\Rule{\IS e {\SG xST}}
     {\IS {\car e} S}
\qquad
\Rule{\IS e {\SG xST}}
     {\IS {\cdr e}{T\sbs{\car e}x} }
\qquad
\begin{array}{l}
\car(s\pr t : \SG xST)\equiv s\\
\cdr(s\pr t : \SG xST)\equiv t\\
\end{array}
\]


\section{Univalence}



\section{The Universe and J}



\section{Symmetric Inversion}

Every path has an inverse, and if we compose a path with its inverse,
we should obtain an identity path. It seems unlikely that we can
engineer this path to be definitionally equal to that given by
reflexivity. We need somehow to obtain the symmetry property
\[
  \green{\mathrm{Inv}}\:(Q\hb S\EQ T) :
  \VP {S\EQ S} {(Q\PC \SYM Q)} {(\IA{} S)}
\]

This seems like a job for J. But it also seems like the thing that
will require the each-way functions to be coherent.


\end{document}