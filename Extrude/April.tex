\documentclass{article}
\usepackage{a4wide}
\usepackage{pig}
\usepackage{upgreek}
\usepackage{stmaryrd}
\usepackage{amssymb}
\ColourEpigram

\newcommand{\F}[1]{\green{\mathsf{#1}}}
\newcommand{\hb}{\!:\!}
\newcommand{\TY}{\blue{\star}}
\newcommand{\ZERO}{\blue{\mathsf{0}}}
\newcommand{\ONE}{\blue{\mathsf{1}}}
\newcommand{\TWO}{\blue{\mathsf{2}}}
\newcommand{\TO}{\mathrel{\blue{\to}}}
\newcommand{\PI}[2]{\blue{(}#1\hb #2\blue{)\to\;}}
\newcommand{\SG}[2]{\blue{(}#1\hb #2\blue{)\,\times\,}}
\newcommand{\WW}[2]{\blue{(}#1\hb #2\blue{)\,\triangleleft\,}}
\newcommand{\PA}[2]{#1\mathbin{\blue{-}}#2}
\newcommand{\PATH}[3]{#2\mathbin{\blue{\langle}#1\blue{\rangle}}#3}
\newcommand{\void}{\red{\ast}}
\newcommand{\ttt}{{\red{\mathsf{1}}}}
\newcommand{\fff}{{\red{\mathsf{0}}}}
\newcommand{\la}[1]{\red{\uplambda}#1.\,}
\newcommand{\pr}{\red{,}\,}
\newcommand{\tr}{\red{\blacktriangleleft}\,}
\newcommand{\pa}[1]{\red{\uppi}#1.\,}
\newcommand{\el}[1]{\red{\underline{\black{#1}}}}
\newcommand{\fst}{\:\green{\mathsf{fst}}}
\newcommand{\snd}{\:\green{\mathsf{snd}}}
\newcommand{\bad}[1]{\:\green{\mathsf{bad}[}#1\green{]}}
\newcommand{\cond}[4]{\:\green{\mathsf{cond}[}#1.\,#2\green{\,\mid\,}#3\green{\,\mid\,}#4\green{]}}
\newcommand{\ind}[3]{\:\green{\mathsf{rec}[}#1.\,#2\green{\,\mid\,}#3\green{]}}
\newcommand{\ze}{{\orange{\mathsf{0}}}}
\newcommand{\un}{{\orange{\mathsf{1}}}}
\newcommand{\base}{{\!\orange{\textvisiblespace}\!}}
\newcommand{\mux}[3]{#1\orange{(}#2\orange{,}\,#3\orange{)}}
\newcommand{\eval}[1]{\llbracket #1 \rrbracket}
\newcommand{\pj}{\,\orange{@}}
\newcommand{\EQUIV}[7]{\blue{\left[\black{\begin{array}{rcl}
   #1&#7&#2\\
   #3&#5\quad#6&#4\\
   \end{array}}\right]}}
\newcommand{\KAN}[6]{\blue{\begin{array}{|@{}r@{}c@{}l@{}|}
                        & \black{#6} & \\
                     \black{#2} & \black{.#1.} & \black{#3} \\
                        & \black{#4.\,#5} & \\
                      \hline
                     \end{array}}}
\newcommand{\kan}[6]{\red{\begin{array}{|@{}r@{}c@{}l@{}|}
                        & \black{#6} & \\
                     \black{#2} & \black{.#1.} & \black{#3} \\
                        & \black{#4.\,#5} & \\
                      \hline
                     \end{array}}}
\newcommand{\ikan}[7]{\red{\begin{array}{|@{}r@{}c@{}l@{}|}
                        \hline
                     \black{#2} & \black{.#1.} & \black{#3} \\
                        \red{\blacksquare}\hfill & \black{#4.\,#5} & \hfill\red{\blacksquare} \\
                      \hline
                     \end{array}}(#6,#7)}
\newcommand{\hkan}[7]{\green{\begin{array}{@{}r|@{}c@{}|l@{}}
                        & \black{#7} & \\
                        & \black{#1} &\\
                     \black{#3} & \black{.#2.} & \black{#4} \\
                        \cline{2-2}
               \multicolumn{1}{r}{}         & \multicolumn{1}{c}{\black{#5.\,#6}} & \\
                     \end{array}}}
\newcommand{\TYPE}[1]{\TY\ni #1}
\newcommand{\POINT}[1]{\textsc{point}\;#1}

\begin{document}
\title{April Fool}
\author{The Strathcube Gang}
\maketitle

\section{Introduction}

This is another attempt at a `European' cubical type theory which computes.
In this attempt, the type formation judgment makes a reappearance.


\section{Plain Type Theory}

\[\begin{array}{rrll}
S,T,s,t & ::= & \TY   & \mbox{type of types} \\
        &   | & \ZERO & \mbox{empty type}\\
        &   | & \ONE  \;|\; \void & \mbox{unit type and value}\\
        &   | & \TWO \;|\; \fff \;|\; \ttt & \mbox{pair type and values} \\
        &   | & \PI x S T(x)   & \mbox{function types}\\
        &   | & \la x t(x)     & \mbox{abstraction}\\
        &   | & \SG x S T(x)   & \mbox{pair types}\\
        &   | & s \pr t        & \mbox{pair} \\
        &   | & \WW x S T(x)   & \mbox{tree types} \\
        &   | & s \tr t        & \mbox{tree} \\
        &   | & \el e          & \mbox{elimination}
\medskip\\
e,f,E,F & ::= & x              & \mbox{variable}\\
        &   | & e\:a           & \mbox{action}\\
        &   | & s:S            & \mbox{type ascription}
\medskip\\
a       & ::= & \bad T        & \mbox{absurdity elimination}\\
        &   | & \cond x{T(x)}{t_\fff}{t_\ttt}    & \mbox{dependent conditional}\\
        &   | & s           & \mbox{function application}\\
        &   | & \fst          & \mbox{left projection from pair}\\
        &   | & \snd          & \mbox{right projection from pair}\\
        &   | & \ind x{T(x)}t    & \mbox{induction}\\
\medskip\\
\Gamma & ::= & \varepsilon   & \mbox{empty} \\
       &   | & \Gamma,\gamma & \mbox{extension}
\medskip\\
\gamma & ::= & x\hb S & \mbox{variable extension} \\
\end{array}\]


Our habit is to suppress the context that is shared by rules, assuming
its omnipresence. We write $\gamma\vdash A$ for the assertion that $A$
holds after extending the context with $\gamma$.
We write $\dashv \gamma$ for the assertion that the context contains
the given extension $\gamma$.

\subsection{Administration}
\[
\Rule{\dashv x\hb S}
     {x\in S}
\qquad
\Rule{e\in S\quad \TYPE{S\equiv T}}
     {T\ni\el e}
\qquad
\Rule{\TYPE S\quad S\ni s}
     {s:S \in S}
\qquad
\el{t:T}\leadsto t
\qquad
\el{x\:\vec{a}}:T\leadsto x\:\vec{a}
\]

\subsection{Type in Type}
\[
\Axiom{\TYPE\TY}
\]

\subsection{Base Types}
\[
\Axiom{\TYPE\ZERO}
\qquad
\Rule{e\in\ZERO\quad\TYPE T}
     {e\bad T \in T}
\]
\[
\Axiom{\TYPE\ONE}
\qquad
\Axiom{\ONE\ni\void}
\qquad
\Axiom{\ONE\ni t\equiv\void}
\]
We write $x\vDash A(x)$ for the conjunction of $A(\fff)$ and
$A(\ttt)$.

\[
\Axiom{\TYPE\TWO}
\qquad
\Axiom{x\vDash\TWO\ni x}
\qquad
\Rule{e\in\TWO\quad
      x\hb\TWO\vdash \TYPE{T(x)}\quad
      x\vDash T(x)\ni t_x}
     {e\cond x{T(x)}{t_\fff}{t_\ttt}\in T(e)}
\]

Computation:
\[
j\vDash(j:\TWO)\cond x{T(x)}{t_\fff}{t_\ttt}\in T(x) \leadsto t_j:T(j)
\]

\subsection{Function Types}

\[
\Rule{\TYPE S\quad x\hb S \vdash \TYPE T(x)}
     {\TYPE \PI x S T(x)}
\qquad 
\Rule{x\hb S \vdash T(x)\ni t(x)}
     {\PI x S T(x) \ni \la x t(x)}
\qquad
\Rule{f\in \PI x S T(x)\quad S\ni s}
     {f\:s\in T(s:S)}
\]
Computation is as usual:
\[
(\la x t(x) : \PI x S T(x))\:s \leadsto t(s:S) : T(s:S)
\]
Definitional equality includes the $\eta$-rule:
\[
\Axiom{\PI x S T(x) \ni \el f \equiv \la x \el{f\:x}}
\]

\subsection{Pair Types}
\[
\Rule{\TYPE S\quad x\hb S \vdash \TYPE T(x)}
     {\TYPE \SG x S T(x)}
\qquad 
\Rule{S\ni s\quad T(s:S)\ni t}
     {\SG x S T(x) \ni s\pr t}
\qquad
\Rule{e\in \SG x S T(x)}
     {e\fst\in S}
\qquad
\Rule{e\in \SG x S T(x)}
     {e\snd\in T(e\fst)}
\]
Again, we compute as usual:
\[
(s\pr t : \SG x S T(x))\fst\leadsto s:S \qquad 
(s\pr t : \SG x S T(x))\snd\leadsto t:T(s:S)
\]
Again, we have an $\eta$-rule:
\[
\Axiom{\SG x S T(x) \ni \el e \equiv e\fst \pr e\snd}
\]

\subsection{Tree Types}

\[
\Rule{\TYPE S\quad x\hb S \vdash \TYPE T(x)}
     {\TYPE \WW x S T(x)}
\qquad 
\Rule{S\ni s\quad T(s:S) \TO \WW x S T(x) \ni t}
     {\WW x S T(x) \ni s\tr t}
\]
\[
\Rule{\begin{array}{@{}l@{}
     }e\in \WW x S T(x)\quad
      w\hb \WW x S T(x) \vdash \TYPE T'(w) \\
      \PI xS\PI w{T(x) \TO \WW x S T(x)}(\PI
        y{T(x)}T'(w\:\el y))\TO
        T'(\el x\tr \el w:\WW x S T(x)) \ni t'
      \end{array}}
     {e\ind x{T'(x)}{t'}\in T'(e)}
\]
Computation:
\[\begin{array}{@{}l@{}}
(s\tr t : \WW x S T(x))\ind x{T'(x)}{t'}\\
\leadsto
(t':\PI xS\PI w{T(x) \TO \WW x S T(x)}(\PI
        y{T(x)}T'(w\:\el y))\TO
        T'(\el x\tr \el w:\WW x S T(x)))\\
\qquad
\:s\:t\:\la y (t :T(s:S) \TO \WW x S T(x))\:y\ind x{T'(x)}{t'}
\end{array}\]


\section{Points, Path Types and Transportation}


\[\begin{array}{rrll}
S,T,s,t & ::= & \PATH T{t_\ze}{t_\un}  & \mbox{path type}\\
        &   | & \pa i T        & \mbox{path} \\
\medskip\\
a       & ::= & \pj m          & \mbox{path projection}\\
\medskip\\
m       & ::= & i              & \mbox{coordinate} \\
        &   | & \ze            & \mbox{left end} \\
        &   | & \un            & \mbox{right end} \\
        &   | & \mux{i}{m_0}{m_1} & \mbox{rescaling} 
\medskip\\
\Gamma & ::= & \Gamma,i      & \mbox{coordinate extension} \\
\end{array}\]

\subsection{Points}
We write $i\vDash A(i)$ for the conjunction of $A(\ze)$ and
$A(\un)$.

\[
\Rule{\POINT m}
     {\gamma\vdash\POINT m}
\qquad
\Axiom{\POINT \ze}
\qquad
\Axiom{\POINT \un}
\qquad
\Rule{i\vDash\POINT m_i}
     {i\vdash\POINT \mux i{m_\ze}{m_\un}}\;
  m_\ze\neq m_\un
\]


\newcommand{\pop}[1]{#1^{\ze}}

That is, points are reduced ordered binary decision diagrams.

We typically write $i$ for $\mux i \ze\un$ and $\pop i$ for
$\mux i\un\ze$. We may also write $i^{\un}$ for $i$.

We can define hereditary substitution for points because
the operation $\mux m{m_\ze}{m_\un}$ is admissible in general.
The resulting theory validates
\[
i\vDash \mux i{m_\ze}{m_\un}\equiv m_i
\qquad
\mux \_ mm \equiv m
\]


\subsection{Path Types}

\[
\Rule{\TYPE T \quad i\vDash T\ni t_i}
     {\TYPE \PATH{T}{t_\ze}{t_{\un}}}
\qquad
\Rule{i\vdash T\ni t(i)\quad i\vDash T\ni t(i)\equiv t_i}
     {\PATH{T}{t_\ze}{t_\un}\ni \pa i t(i)}
\qquad
\Rule{e\in \PATH{T}{t_\ze}{t_\un}\quad \POINT m}
     {e\pj m \in T}
\]
We like to write $\PA{T_\ze}{T_\un}$ for $\PATH\TY{T_\ze}{T_\un}$. These paths between
types are equipped with another eliminator:
transportation overloads application, treating type paths as explicit coercions.
\[
\Rule{E\in \PA{T_\ze}{T_\un}\quad T_\ze\ni t_\ze}
     {E\:t_\ze\in T_\un}
\]
Computation for projection (which relies on type reconstruction):
\[
i\vDash \Rule{e\in \PATH{T}{t_\ze}{t_\un}}
{E\pj i \leadsto t_i}
\qquad
\Axiom{(\pa i t(i) : \PATH{T}{t_\ze}{t_\un}) \pj m \leadsto t(m):T}
\]
Computation for transportation is a bigger deal which we'll get back
to. But we will have at least the fact that going nowhere does
nothing. Thierry calls this `regularity', which is indeed an
admirable quality of a transportation system.
\[
(\pa\_ T : \PA TT)\:t \leadsto t:T
\]


\section{Kan Constructions}

A Kan construction is at some point $m$ in the top of a square, where we know
both sides and the base. The Kan types thus provide the means to fill
in the top of the square.

\[\begin{array}{rrll}
S,T,s,t     & ::= & \kan i{t_\ze(i)}{t_\un(i)}j{t_\base(j)}m & \mbox{Kan construction}
\end{array}\]

Kan constructions represent interpolant values in the top
of a square, given values at the sides and the base.
\[
\Rule{j\vDash i\vdash T\ni t_j(i) \quad
      j\vdash T \ni t_\base(j) \quad
      j\vDash T \ni t_j(\ze)\equiv t_\base(j) \quad
     \POINT m}
     {T \ni
      \kan i{t_\ze(i)}{t_\un(i)}j{t_\base(j)}m}
\]

When $T$ is $\TY$, we like to draw the three sides in blue, but it's
not a new thing, just a special case.
\[
\Rule{j\vDash i\vdash \TYPE T_j(i) \quad
      j\vdash \TYPE S_\base(j) \quad
      j\vDash \TYPE T_j(\ze)\equiv T(j) \quad
     \POINT m}
     {\TYPE
      \KAN i{T_\ze(i)}{T_\un(i)}j{S_\base(j)}m}
\]


Kan types collapse in various special cases:
\begin{description}
\item[bumping the sides] If the coordinate $m$ is either $\ze$ or
  $\un$, the Kan object collapses to its relevant top corner.
\[
j\vDash\begin{array}{l}
\kan i{t_\ze(i)}{t_\un(i)}j{t_\base(j)}j \leadsto t_j(\un)
\end{array}
\]
\item[flattening to its base] If both sides don't go anywhere, the
  whole square (hence its top) is the degenerate base.
\[
\begin{array}{l}
\kan \_{t_\base(\ze)}{t_\base(\un)}j{t_\base(j)}m \leadsto t_\base(m)
\end{array}
\]
\end{description}

Note, however, that tall thin Kan constructions do \emph{not}
collapse.
\[
\kan i{t(i)}{t(i)}\_{t(\ze)}m \not\leadsto t(\un)
\]

Rescaling means we can construct the whole square. We define
\[
\ikan i{t_\ze(i)}{t_\un(i)}j{t_\base(j)}ij =
\kan k{t_\ze(\mux k\ze i)}{t_\un(\mux k\ze i)}j{t_\base(j)}j
\]
and we note that the collapse rules make the square coincide with the
known points on its boundary.
\[
\ikan i{t_\ze(i)}{t_\un(i)}j{t_\base(j)}\ze j \equiv t_\base(j)
\qquad
j\vDash \ikan i{t_\ze(i)}{t_\un(i)}j{t_\base(j)}ij \equiv t_j(i)
\]
Moreover, the construction ensures that the lid also fits.
\[
\ikan i{t_\ze(i)}{t_\un(i)}j{t_\base(j)}\un j \equiv \kan i{t_\ze(i)}{t_\un(i)}j{t_\base(j)}j
\]

If we have a square of types, we can define the \emph{heterogeneous}
Kan construction.
\[
\Rule{i\vdash j\vdash \TYPE T(i,j) \qquad
      j\vDash i\vdash T(i,j)\ni t_j(i) \qquad
      j\vdash T(\ze,j)\ni t_\base(j) \qquad
      j\vDash T(\ze,j)\ni t_j(\ze)\equiv t_\base(j) \qquad
      \POINT m}
   {\hkan{i.j.T(i,j)}i{t_\ze(i)}{t_\un(i)}j{t_\base(j)}m\in T(\un,m)}
\]
The square of types allows us, by transportation, to iron the values
down into whichever type from the square we happen to want
\[
\hkan{i.j.T(i,j)}i{t_\ze(i)}{t_\un(i)}j{t_\base(j)}m\in T(\un,m) =
  \kan i{(\pa k{T(\mux k{(i,\ze)}{(\un,m)}})\:t_\ze(i)\hspace*{-0.8in}}
        {\hspace*{-0.8in}(\pa k{T(\mux k{(i,\un)}{(\un,m)})})\:t_\un(i)}
       j{(\pa k {T(\mux k{(\ze,j)}{(\un,m)})})\:t_\base(j)}m
  : T(\un,m)
\]
writing $\mux k{(m_\ze,m'_\ze)}{(m_\un,m'_\un)}$ for
$(\mux k {m_\ze}{m_\un},\mux k {m'_\ze}{m'_\un})$.

Doubtless, there'll be more to say.


\section{Transportation}

How do you send something over a type path? It depends on the path,
but at least rescaling keeps things coherent. Whenever we have
\[
\pa i S(i) : \PA {S(\ze)}{S(\un)}
\]
it will be convenient to define
\[
S(j-k) = \pa i S(\mux ijk) : \PA {S(j)}{S(k)}
\]
allowing us to transmit values from any entry point on the original
path to any exit point, not just from $\ze$ to $\un$.


\subsection{Function, Pair and Tree Transportation}

Transporting pairs works componentwise.

\[
(\pa i \SG x {S(i)}{T(i,x)})\:p_\ze \leadsto
  s(\un) \pr (\pa i {T(i,s(i))})\:(p_\ze\snd)
  \quad\mbox{where}\;s(i) = S(\ze-i)\:(p_\ze\fst)
\]
We need to be able to move the first component anywhere in order
to build the path for transporting the second component. Rescaling
lets us do that. If the pair type does not actually depend on $i$,
then neither will the path $S(\ze,i)$, so $s(i)$ will be a constant,
and the path transporting the second component will also be
degenerate. Hence, this rule is compatible with regularity.

Transporting functions amounts to transporting the input from $\un$
to $\ze$, then the output back to $\un$.
\[
(\pa i \PI x {S(i)}{T(i,x)})\:f_\ze \leadsto
  \la {s_\un} (\pa i {T(i, s(i))})\:(f_\ze\:s(\ze))\quad
  \mbox{where}\;s(i) = S(\un-i)\:s_\un
\]
Note that $s(\un)\equiv s_\un$, so we have the correct type. Moreover,
if our function type does not depend on $i$, then $s(\_)=s_\un$,
and $T(i,s(i))$ does not depend on $i$ either. Hence, this rule
is compatible with regularity.

Transporting trees is a similar dance, but strict.
\[\begin{array}{l}
(\pa i \WW x {S(i)}{T(i,x)})\:(s_\ze\tr t_\ze) \leadsto
  s(\un)\tr \la {x_\un} (\pa i \WW x
  {S(i)}{T(i,x)})\:(t_\ze\:(T'(\un,\ze)\:x_\un))
\\ \qquad\mbox{where}\; s(i)=S(\ze-i)\:s_\ze;\;
  T'(i) = T(i, s(i))
\end{array}\]


\subsection{Path Transportation}

Transporting paths requires us to build the lid of a square when we
know both sides and the base. That's why we invented Kan
constructions.
\[
(\pa i \PATH{T(i)}{t_\ze(i)}{t_\un(i)})\:t_\base \leadsto
\hkan{i.j.T(i)}i{t_\ze(i)}{t_\un(i)}j{t_\base\pj j}{-}
\]
If the path type does not depend on $i$, neither do the sides of the
square, so it collapses to its base. Hence, this rule is compatible
with regularity.


\subsection{Kansportation}

How on earth do we transport values between Kan types?
\[
\left(\pa h \KAN i{T_\ze(h,i)}{T_\un(h,i)}j{T_\base(h,j)}{m(h)}\right)\:t \leadsto \;?
\]

The need to be compatible with regularity means we should not take
values down to the base and up again unless we are sure the path is
not degenerate. And the only way that can happen is if we can see
that the path is \emph{diagonal}. That is the situation given by
\[
k\vDash \left(\pa h \KAN i{T_\ze(h,i)}{T_\un(h,i)}j{T_\base(h,j)}{h^k}\right)\:t \leadsto \;?
\]
where we are transporting from the point with $(h,i,j)$ coordinates
$(\ze,\un,k^\ze)$ to $(\un,\un,k)$. We transport down to the base,
across the diagonal of the base, then up.
\[
 (T_{k}(\un,(\ze,\un)) \cdot (\pa h T_\base(h,h^k)) \cdot T_{k^\ze}(\ze,(\un,\ze)))\:t
\]
This rule is trivially compatible with regularity, because there is
only one thing that can make such a path degenerate: $h^k$ cannot be constant
in $h$, so degeneration must be caused by the sides of the squares
collapsing to a degenerate base, in which case all three stages of our
transportation become trivial.

Meanwhile, if $m(\ze)=m(\un)$ then $m(h)$ is a constant, as a
consequence of our affine language of points. If this $m$ is a
canonical constant, then the square collapses to one corner for every
$h$, so transportation over the square reduces to transportation along
one `corner rail': in fact, if we collapse aggressively, we will not
even see such problems.

That is, the diagonal rule is all we need.


\section{Kan Distributes Over Elimination}

The Kan construction invents a great many seemingly `new' elements of
old types. We had better equip these new values with behaviour. To see
the necessity, consider a Kan construction $F(i)\in A \TO \TY$: how do we
transport along the edge of $\pa i F(i)\:a$ unless each application
$F(i)\:A$ reduces to to a Kan construction?

Fortunately, this is not too tricky to arrange.
\[
\left(\kan i{f_\ze(i)}{f_\un(i)}j{f_\base(j)}m:\PI xST(x)\right)\:s \leadsto
\kan i{(f_\ze(i):\PI xST(x))\:s\hspace*{-0.7in}}
      {\hspace*{-0.7in}(f_\un(i):\PI xST(x))\:s}
     j{(f_\base(j):\PI xST(x))\:s}m : T(s:S)
\]

First projection is just as easy
\[
\left(\kan i{p_\ze(i)}{p_\un(i)}j{p_\base(j)}m:\SG xST(x)\right)\fst \leadsto
\kan i{(p_\ze(i):\SG xST(x))\fst\hspace*{-0.7in}}
      {\hspace*{-0.7in}(p_\un(i):\SG xST(x))\fst}
     j{(p_\base(j):\SG xST(x))\fst}m : S
\]
but second projection requires a little type repair.
\[\begin{array}{l}
\left(\kan i{p_\ze(i)}{p_\un(i)}j{p_\base(j)}m:\SG xST(x)\right)\snd
    \leadsto \\ \qquad
  \hkan {i.j.T(s(i,j))} i{(p_\ze(i):\SG xST(x))\snd}
      {(p_\un(i):\SG xST(x))\snd}
     j{(p_\base(j):\SG xST(x))\snd}m \\
\qquad\qquad  : T(s(\un,m)) \\
\qquad  \mbox{where}\quad s(i,j) =
    \left(\ikan i{p_\ze(i)}{p_\un(i)}j{p_\base(j)}ij:\SG xST(x)\right)\fst
\end{array}\]

The general situation is that if we have
\[
\Rule{e\in T\quad\cdots}
     {e\:a\in P(e)}
\]
then
\[\begin{array}{l}
\left(\kan i{t_\ze(i)}{t_\un(i)}j{t_\base(j)}m:T\right)\:a
\leadsto 
\hkan {i.j.P(t(i,j):T)}
   i{(t_\ze(i):T)\:a}{(t_\un(i):T)\:a}
   j{(t_\base(j):T)\:a}m
    : P(t(\un,m))
\\ \qquad
  \mbox{where}\quad t(i,j) = \ikan i{t_\ze(i)}{t_\un(i)}j{t_\base(j)}ij
\end{array}\]


\section{Should Kan Distribute Over Introduction?}

It doesn't seem to be necessary for canonicity, but we could consider
rules which allow `smoothing' of Kan squares. For example, if we have
a Kan type where the sides and base are all function types
\[
\KAN i{\PI x{S_\ze(i)}T_\ze(i,x)\hspace*{-0.5in}}
      {\hspace*{-0.5in}\PI x{S_\un(i)}T_\un(i,x)}
     j{\PI x{S_\base(i)}T_\base(i,x)}m
\]
it might be reasonable to presume that the whole square might consist
entirely of function types, and hence, in particular, the type at
point $m$ in the lid. Might this become
\[\begin{array}{l}
\PI x{\KAN i{S_\ze(i)}{S_\un(i)}j{S_\base(j)}m}
     {\KAN i{T_\ze(s(i,\ze))}{T_\un(s(i,\un))}j{T_\base(s(\ze,j))}m}
\smallskip\\ \qquad \mbox{where}\quad
  s(i,j) = \left(\pa k \ikan
    i{S_\ze(i)}{S_\un(i)}j{S_\base(j)}{\mux k\un i}{\mux kmj}\right)\:x
\end{array}\]
?



\section{J and Regularity}

We have contractibility of singletons
\[\begin{array}{rcl}
\F{one}(X:\TY)(x:X)(y:X)(q:\PATH Xxy) & : &
  \PATH{\SG yX\PATH Xxy}{(x\pr\pa\_x)}{(x\pr q)} \\
&=& \pa i (q\pj i\pr q\pj{(\ze-i)})
\end{array}\]
and its right-ended counterpart
\[\begin{array}{rcl}
\F{eno}(X:\TY)(x:X)(y:X)(q:\PATH Xxy) & : &
  \PATH{\SG xX\PATH Xxy}{(y\pr\pa\_x)}{(y\pr q)} \\
&=& \pa i (q\pj i^\ze\pr q\pj{(i^\ze-\un)})
\end{array}\]

As a result, we have
\[\begin{array}{rcl}
\F{J}(X:\TY)(x:X)(y:X)(q:\PATH Xxy)(P:(\SG yX\PATH Xxy)\TO\TY)
   (p:P\:(x\pr\pa\_x))  & : & P\:(y\pr q) \\
 = \quad (\pa i P\:(\F{one}\:X\:x\:y\:q \pj i))\:p
\end{array}\]
and our rule for transporting across constant functions gives us that
if $q$ is constant, $\F{J}$'s transport path is constant, so we get exactly $p$.


\section{Univalence}

We treat univalence by introducing a further notion of interpolant
type.
\[
\Rule{\begin{array}{@{}l@{}}
      i\vDash\TYPE T_i\qquad
      i\vDash T_i\TO T_{i^\ze}\ni f_i \qquad
      \PI{x_\un}{T_\un}\PATH{T_\un}{f_\ze\:(f_\un\:x_\un)}{x_\un} \ni
        q \qquad \POINT m \\
      \PI{x_\ze}{T_\ze}\PI{x_\un}{T_\un}
         \PI{q'}{\PATH{T_\un}{f_\ze\:x_\ze}{x_\un}}
          \PATH{\SG{x_\ze}{T_\ze}\PATH{T_\un}{f_\ze\:x_\ze}{x_\un}}
               {(f_\un\:x_\un \pr q\:x_\un)}{(x_\ze\pr q')}
        \ni u 
     \end{array}}
     {\TYPE \EQUIV{T_\ze}{T_\un}{f_\ze}{f_\un}qum}
\]
That is, the $f_i$ give us an isomorphism between the $T_i$.
The $q$ tells us that $f_\un$ computes the inverse image of $f_\ze$
at any given $t_\un$, giving us the path that connects them.
The $u$ tells us that $f_\un$ and $q$ are unique for $f_\ze$.
Note that
\[
  \PATH{T_\ze}{f_\un\:(f_\ze\:x_\ze)}{x_\ze}
  \ni \pa i u\:x_\ze\:(f_\ze\:x_\ze)\:(\pa\_f_\ze\:x_\ze)\pj i\fst
\]
so we do round-trip both ways.

As with Kan constructions, these interpolants collapse under certain
circumstances. Certainly, they connect to their endpoints.
\[
i\vDash \EQUIV{T_\ze}{T_\un}{f_\ze}{f_\un}qui
  \leadsto T_i
\]
They also collapse when the isomorphism is trivial.
\[
\EQUIV{T}{T}{\la xx}{\la xx}{\la x\pa\_x}
  {\F{eno}\:T}m \leadsto T
\]
\textbf{(Do we care about the proofs when collapsing trivial isos? Perhaps not.)}

Transportation across interpolants is triggered only when there can
be no conflict with regularity, i.e., from some canonical $j$ to $j^\ze$.
\[
j\vDash (\pa i
\EQUIV{T_\ze(i)}{T_\un(i)}{f_\ze(i)}{f_\un(i)}qu{i^{j^\ze}})\:x_j
   \leadsto f_j(\un)\:(T_j(\ze-\un)\:x_j)
\]
The choice to deploy the isomorphism at $\un$ \emph{after} transportation is
arbitrary, but the result is (\textbf{check}!) provably the same as
transporting after deploying the isomorphism at $\ze$.

Note that the proofs play no part in the transportation mechanism. So
what are they for? We can use $\F{J}$ to construct a \emph{general} proof
that transportation there-and-back across a type path
round-trips. Plugging in univalence will give us an
alternative candidate for the $q$ component, but $u$ tells us that any
such alternative is related to $q$.




\end{document}