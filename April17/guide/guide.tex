\documentclass{article}
\usepackage{a4}
\usepackage{alltt}

\begin{document}

\title{the {\tt platypus} user guide}
\author{team platypus}
\maketitle


\section{introduction}

The {\tt platypus} system does not yet exist, but let's write its user guide anyway,
as that will help us figure out what \emph{should} exist.

Platypus is a rudimentary proof assistant based on constructive type theory, in the
tradition of the constructive engine. It is
\begin{description}
\item[command-driven] using all the user interface technologies of a thoroughly
  modern paper teletype,
\item[inconsistent] thanks to type in type,
\item[European] in the sense that the theory has a notion of \emph{definitional}
  equality which is computed, and thus a disappointing accident,
\item[ugly] because its syntax is not exactly s-expressions but in that area
\end{description}
and it is all of these things not because they are virtues, but because they are
cheap.


\section{the syntax of everything}

We have a universal syntax.
\[\begin{array}{rrll}
\mathit{thing} & ::= & \mathit{atom}
  & \mbox{nonempty sequence of nonspace chars bar {\tt .,()}}\\
               &   | & \mathit{atom}.\,\mathit{thing}
  & \mbox{binding}\\
               &   | & \mathit{thing}\;\mathit{thing}+
  & \mbox{clump of two or more things} \\
               &   | & {\tt (}\mathit{thing}{\tt )}
  & \mbox{parentheses for disambiguation} \\
\end{array}\]
You \emph{may} insert parentheses for disambiguation anywhere. You
\emph{must} use them to wrap a binding or a clump which is not the
last thing in another clump. You need not wrap in parentheses a
binding that stands last in a clump. So
\[
  f\;x.\,a\; b\;\;\mbox{means}\;\;(f\;(x.\,(a\; b))) \qquad
  f\;(x.\,a)\; b\;\;\mbox{means}\;\;(f\;(x.\,a)\; b)
\]
If a clump stands last in a clump, you may omit the parentheses
around the inner clump if you insert a {\tt ,} before it. So
\[
 a\;b\;c,\,d\;e\;f\;\;\mbox{means}\;\;(a\;b\;c\;(d\;e\;f))
\]
In this way, as in the 1950s, right-nested things flatten into
list-like things. We will also be predisposed to make binding
constructs put their binding last. E.g., dependent function types
look like
\[
  {\tt Pi}\;S\;x.\,T
\]
and a big long telescope of such things is
\[
  {\tt Pi}\;S_0\;x_0.\,{\tt Pi}\;S_1\;x_1.\,\cdots {\tt Pi}\;S_n\;x_n.\, T
\]
Degenerate vacuous binding is indicated by \emph{not binding}, rather
than binding an underscore or some such. So for a nondependent
function
type, we have
\[
  {\tt Pi}\;S_0,\,{\tt Pi}\;S_1,\,\cdots {\tt Pi}\;S_n\; T
\]

We use this syntax for everything, including commands typed in the
shell. You may not type a syntax error (but, of course, you may make
use of this syntax in meaningless ways). When you hit enter, your
input so far is presumed complete if it parses. If, however, the
parser still awaits more text (usually a {\tt )}) to form valid input,
you will be prompted to supply it. So, you can give a multi-line
clump as a command if you take the precaution to begin it with {\tt (}.


\section{the business is definition}

The point of a {\tt platypus} session is to construct some \emph{definitions} by refinement,
with intermediate unknowns represented as \emph{holes}.

A definition has
\begin{description}
\item[a name] which is some sort of identifier, consisting of nonwhitespace
  characters other than {\tt (),.}; by convention the character {\tt /} is
  used to give names a hierarchical structure, so {\tt foo/goo} is the name
  of {\tt foo}'s {\tt goo};
\item[a telescope of parameters] which is a type theoretic context; uses of
  the definition must instantiate the parameters;
\item[a type];
\item[a value] which must have the indicated type in the scope of the
  parameters; values may refer to \emph{holes} if the definition is incomplete
\end{description}

A hole is like a definition, except that its name ends with {\tt ?} and its value is unknown. As soon as the hole is filled, its usage sites are refined and the hole ceases to exist.

Although names are suggestive of hierarchical structure, we flatten the
hierarchy into a \emph{state}, which is a sequence of holes and definitions.

What is the syntax of a definition?
\[
  \mathit{name}{\tt ,}\;\mathit{body}
\]
where
\[\begin{array}{rrl}
\mathit{body} & ::= & {\tt =}\;\mathit{term}\;{\tt :}\;\mathit{type} \\
              &   | & \mathit{type}\;x.\mathit{body} \\
\end{array}\]
E.g., one might define a useless identity operation, named {\tt foo} as follows.
\begin{verbatim}
  foo, Type X. X x. = x : X
\end{verbatim}

One can, of course, imagine a nicer syntax. (A body is always a
\emph{clump}, hence the comma.)


\section{where am I and what can see from there?}

The command prompt looks like
\[
  \mathit{prefix}\;\mathit{range}\;{\tt>}
\]
where a $\mathit{prefix}$ is either empty or an identifier (typically
a pathlike sequence of ${\tt /}\mathit{name}$ blocks), and a
$\mathit{range}$ is either empty or
\[
  {\tt (}[\mathit{nameA}]\;\mbox{\tt \^{}}\;[\mathit{nameB}]{\tt )}
\]
where either but not both $\mathit{nameX}$ may be omitted.
The prompt tells us our \emph{field of view}: we can see everything whose name
has $\mathit{prefix}$ as a prefix and which comes strictly between $\mathit{prefix}{\tt /}\mathit{nameA}$ and $\mathit{prefix}{\tt /}\mathit{nameB}$. If the prefix is nonempty, there should be something whose name begins with it. Likewise the range endpoints should be
names of actual things.

Before the prompt is issued, you will see a display of all the things in the field of view. The bodies of definitions whose names are strictly longer than the prefix are suppressed.
The display will give only the suffix of the name, to save space.
A hole (whose name is $\mathit{prefix}{\tt /}\mathit{suffix}{\tt ?}$is
displayed as a vertical context, followed by a line of dashes, followed by
the name $\mathit{suffix}{\tt ?}$, a {\tt :} and the hole's type. E.g.,
\begin{verbatim}
  X : Type
  x : X
----------------
  id/body? : X

>
\end{verbatim}

A definition whose name is strictly longer than the prefix will have
its value suppressed, but its context and type will be displayed in
the same format as a goal (but with no {\tt ?}). A definition whose
name is exactly the prefix will have 
If the prefix is $\mathit{blah}$ for some definition $\mathit{blah}$, the body
of the definition will be displayed, instead of the (empty)name
suffix. We might hope to arrive at the following
\begin{verbatim}
  X : Type
  x : X
--------------------------------------
  (\, \x . x) : Pi Type X. Pi X x. X
\end{verbatim}


Some time earlier, you might have typed at the prompt {\tt >}
\begin{verbatim}
> id, = ? : Pi Type X. Pi X x. X
\end{verbatim}
which would result in the display.
\begin{verbatim}
------------------------------
id? : Pi Type X. Pi X x. X

------------------------------
id : Pi Type X. Pi X x. X

>
\end{verbatim}
and you might then have typed
\begin{verbatim}
> = \ X. \ x. ?body
\end{verbatim}
which would result in the goal display above.

Your primary means of not getting bombarded with too much information is thus
to lengthen the prompt. The command
\[
{\tt /}\;\mathit{suffix}\;
\]
will add ${\tt /}\mathit{suffix}$ to the prompt prefix. It is forbidden
to have a nonempty prefix which refers to nothing in the context.

Meanwhile, the command
\[
{\tt <}
\]
will shorten the prompt to its penultimate {\tt /}, or to empty if there is at most
one {\tt /}.

The command `zoom out to $\mathit{name}$'
\[
{\tt <}\;\mathit{name}
\]
will change the prefix to its longest proper subsequence ending in
${\tt /}\mathit{name}{\tt /}$
if there is such a thing.

The command {\tt <<} zooms all the way out to the root.

Your secondary means of not being bombarded with too much information is to narrow
the range with the command
\[
\mathit{name}\;\mbox{\tt \^{}}\;\mathit{name}'
\]
which means `show me only the things between the two names'. You may omit either end
to mean `show me from the beginning' or `show me to the end'. You can use this also to
widen your view. In particular, the command
\[
\mbox{\tt \^{}}
\]
means `show everything'.

But where am I? If I make a definition, where will it go? I am just before
the first hole in my field of view (which means that if there is no such hole, I am at the
end of what I can see. So, viewing range restrictions let us insert definitions
before the end of the state.



\newpage\appendix

\section{Command Cheatsheet}

\[\begin{array}{ll}
{\tt =}\;\mathit{term} & \mbox{give a refinement} \smallskip\\

\mathit{name}\;{\tt /} & \mbox{zoom prefix in to name}\\
{\tt <}\;[\mathit{name}] & \mbox{zoom prefix out [to name]}\\
{\tt <<} & \mbox{zoom prefix all the way out}\\
{}[\mathit{name}]\;\mbox{\tt \^{}}\;[\mathit{name}] & \mbox{set the range}\\
\end{array}\]

\end{document}